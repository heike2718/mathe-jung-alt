\newcommand{\bem}{\par {\bf Bemerkung: }}
\newcommand{\hinw}{\par {\bf Hinweis: }}
\newcommand{\antw}{\par {\bf Antwort: }}
\newcommand{\begr}{\par \rule[-5cm]{0mm}{5cm}{\bf Begründung: }}
\newcommand{\qun}{\par \par \rule{0mm}{5ex}{\bf Quellennachweis:}\par}
\newcommand{\begrit}{{\it Begründung: }}

%Euro, damit hinter dem EURO- Zeichen ein Abstand erscheint
\newcommand{\eu}{\euro \mbox{ }}
\newcommand{\eumath}{\mbox{ \eu }}
\newcommand{\ctmath}{\mbox{ ct }}

% Maßeinheiten für mathematische Formeln
\newcommand{\cm}{\mbox{cm}}
\newcommand{\Cm}{\mbox{ cm}}
\newcommand{\CM}{\mbox{ cm }}
\newcommand{\cM}{\mbox{cm }}
\newcommand{\ggt}{\mbox{ggT}}

%Zahlenrüume
\newcommand{\mathnat}{{\Bbb N}}
\newcommand{\mathrat}{{\Bbb Q}}
\newcommand{\mathganz}{{\Bbb Z}}
\newcommand{\mathreell}{{\Bbb R}}
\newcommand{\mathkomp}{{\Bbb C}}

%zahler zurücksetzen
\newcommand{\resetequation}{\setcounter{equation}{0}}

\newcommand{\antline}[1]{\mbox{\rule{0cm}{1cm}\rule[-2pt]{0.5em}{0em}\rule[-2pt]{#1}{1pt}\rule[-2pt]{0.5em}{0em}}} %parameter ist Länge der Linie.

%  \bild{12cm}{../resources/002/00987.eps}
\newcommand{\bild}[2] {\begin{center}\begin{tabular}{l} \setlength{\epsfxsize}{#1}\epsfbox{#2} \end{tabular}\end{center}}
\newcommand{\leftbild}[2] {\begin{tabular}{l} \setlength{\epsfxsize}{#1}\epsfbox{#2} \end{tabular}}
\newcommand{\graph}[3]{\includegraphics[width=(#1)cm, height=(#2)cm]{#3}}

\newcommand{\bLLletterbox}[4]{\makebox[#1]{\rule[#2]{0em}{#3}\bf\LARGE{#4}}} %\letterbox{breite}{lift}{höhe}{Buchstabe} Text \bf\LARGE
\newcommand{\bLletterbox}[4]{\makebox[#1]{\rule[#2]{0em}{#3}\bf\Large{#4}}} %\letterbox{breite}{lift}{höhe}{Buchstabe} Text \bf\Large
\newcommand{\blletterbox}[4]{\makebox[#1]{\rule[#2]{0em}{#3}\bf\large{#4}}} %\letterbox{breite}{lift}{höhe}{Buchstabe} Text \bf\large
\newcommand{\LLletterbox}[4]{\makebox[#1]{\rule[#2]{0em}{#3}\LARGE{#4}}} %\letterbox{breite}{lift}{höhe}{Buchstabe} Text \LARGE
\newcommand{\Lletterbox}[4]{\makebox[#1]{\rule[#2]{0em}{#3}\Large{#4}}} %\letterbox{breite}{lift}{höhe}{Buchstabe} Text \Large
\newcommand{\bletterbox}[4]{\makebox[#1]{\rule[#2]{0em}{#3}\bf{#4}}} %\letterbox{breite}{lift}{höhe}{Buchstabe} Text \bf
\newcommand{\letterbox}[4]{\makebox[#1]{\rule[#2]{0em}{#3}{#4}}} %\letterbox{breite}{lift}{höhe}{Buchstabe} Text normal

\newcommand{\quelle}[2]{{\small{{\bf #1:} #2}} \newline} % #1: Aufgabenüberschrift #2: Quelle

\newcommand{\vertbox}[1]{\par \makebox[16cm]{\rule{0mm}{#1}}} %weiße vertikale lücke der höhe #1
%\newcommand{\vertbox}[1]{\\[#1]}}

\newcommand{\horbox}[1]{\par \makebox[#1]{\rule{0mm}{#1}}} %weiße horizontale Lücke der breite #1
\newcommand{\horobox}[1]{\makebox[#1]{\rule{0mm}{#1}}} %weiße horizontale Lücke der breite #1 ohne paragraph

\newcommand{\merkbox}[2]{\begin{center}
\fbox{\parbox{#1}{#2}}
\end{center}} %%gerahmte box der breite #1,die Text #2 umschließt

%\newcommand{\kasten}[4]{\rule{#4}{0mm} \fbox{\rule[#3]{#1}{0mm} \rule[-5mm]{0mm}{#2}}  \rule{#1}{0mm}}
%schwarzumrahmter kasten der breite #1 und der höhe #2,unterline #3, abstand zum Nachbarn #4
\newcommand{\kasten}[4]{\rule{#4}{0mm} \fbox{\rule[#3]{#1}{0mm} \rule{0mm}{#2}}  \rule{#1}{0mm}}
%schwarzumrahmter kasten der breite #1 und der höhe #2, underline #3
\newcommand{\hwboxu}[3]{\framebox[#1]{\rule[#3]{0mm}{#2}}}
%schwarzumrahmter kasten der breite #1 und der höhe #2
\newcommand{\hwbox}[2]{\framebox[#1]{\rule{0mm}{#2}}}

%unumrahmter kasten der breite #1 und der höhe #2,unterline #3, abstand zum Nachbarn #4
\newcommand{\kastenblank}[4]{\rule{#4}{0mm} \rule[#3]{#1}{0mm} \rule{0mm}{#2}  \rule{#1}{0mm}}
%unumrahmter kasten der breite #1 und der höhe #2, underline #3
\newcommand{\hwblankboxu[3]}{\rule{#1}{0mm} \rule[#3]{0mm}{#2}}
%unumrahmter kasten der breite #1 und der höhe #2
\newcommand{\hwblankbox[2]}{\rule{#1}{0mm} \rule{0mm}{#2}}

%Anf�hrungsstriche
\newcommand{\anfoben}{\grqq $\,$}
\newcommand{\anfunten}{\glqq}

%Doppelpfeil links rechts 
\newcommand{\gdw}{\Leftrightarrow}
\newcommand{\impl}{\Rightarrow}

%�nderung Formelnumerierung auf Buchstaben
\newcounter{saveeqn}
\newcommand{\alpheqn}{\setcounter{saveeqn}{\value{equation}}}
\setcounter{equation}{0}
\renewcommand{\theequation}{\mbox{\aerabic{saveeqn}-\alph{equation}}}
\newcommand{\reseteqn}{\setcounter{equation}{\value{saveeqn}}}
\renewcommand{\theequation}{\arabic{equation}}

%�nderung font
\newcommand{\changefont}[3]{\fontfamily{#1} \fontseries{#2} \fontshape{#3} \selectfont}

% TikZ-Commands
% Three counters
\newcounter{x}
\newcounter{y}
\newcounter{z}
% The angles of x,y,z-axes
\newcommand\xaxis{210}
\newcommand\yaxis{-30}
\newcommand\zaxis{90}

% The top side of a cube
\newcommand\topside[3]{
  \fill[fill=lightgray, draw=black,shift={(\xaxis:#1)},shift={(\yaxis:#2)},
  shift={(\zaxis:#3)}] (0,0) -- (30:1) -- (0,1) --(150:1)--(0,0);
}

% The left side of a cube
\newcommand\leftside[3]{
  \fill[fill=gray, draw=black,shift={(\xaxis:#1)},shift={(\yaxis:#2)},
  shift={(\zaxis:#3)}] (0,0) -- (0,-1) -- (210:1) --(150:1)--(0,0);
}

% The right side of a cube
\newcommand\rightside[3]{
  \fill[fill=darkgray, draw=black,shift={(\xaxis:#1)},shift={(\yaxis:#2)},
  shift={(\zaxis:#3)}] (0,0) -- (30:1) -- (-30:1) --(0,-1)--(0,0);
}

% The cube 
\newcommand\cube[3]{
  \topside{#1}{#2}{#3} \leftside{#1}{#2}{#3} \rightside{#1}{#2}{#3}
}

% Definition of \planepartition
% To draw the following plane partition, just write \planepartition{ {a, b, c}, {d,e} }.
%  a b c
%  d e
\newcommand\planepartition[1]{
 \setcounter{x}{-1}
  \foreach \a in {#1} {
    \addtocounter{x}{1}
    \setcounter{y}{-1}
    \foreach \b in \a {
      \addtocounter{y}{1}
      \setcounter{z}{-1}
      \foreach \c in {1,...,\b} {
        \addtocounter{z}{1}
        \cube{\value{x}}{\value{y}}{\value{z}}
      }
    }
  }
}

\tikzset{heikes grid/.style={help lines, color=#1, step=.25cm}}
\tikzset{heikes grid/.default=gray}

\renewcommand{\figurename}{Bild}

\DeclareMathOperator{\ggT}{ggT}
\DeclareMathOperator{\kgV}{kgV}

\newcolumntype{L}[1]{>{\raggedright\arraybackslash}p{#1}}
\newcolumntype{C}[1]{>{\centering\arraybackslash}p{#1}}
\newcolumntype{R}[1]{>{\raggedleft\arraybackslash}p{#1}}

% === Farben ===
\definecolor{darkgreen}{RGB}{4,150,19}


